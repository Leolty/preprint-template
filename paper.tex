% ======================================================================
%  Triton Template: Your Complete Guide to Beautiful Academic Preprints
%  Author: Tianyang Liu 
%          - <til040@ucsd.edu>
%          - https://leoii22.com/
% ===========================================================================

\documentclass{triton}

% ----------------------------------------------------------------------
%  External math macro collection (ships with this template)
% ----------------------------------------------------------------------
\input{math_commands.tex}

% ----------------------------------------------------------------------
%  Essential packages for modern academic papers
%  (Some are already included in triton.cls; duplicates are safe.)
% ----------------------------------------------------------------------
\usepackage{amsmath,amsfonts,amssymb,mathtools,bm}  % advanced math
\usepackage{thmtools}          % nicer theorem handling
\usepackage{algorithm}         % algorithm float
\usepackage{algpseudocode}     % algpseudocode commands (\Require,\State…)
\usepackage{enumitem}          % tight lists
\usepackage{siunitx}           % \SI{42}{\kilo\meter\per\second}
\usepackage{tikz}              % inline vector graphics
\usepackage{booktabs}          % professional tables
\usepackage{subcaption}        % subfigures

% ----------------------------------------------------------------------
%  Theorem-like environments for your papers
% ----------------------------------------------------------------------
\declaretheorem[name=Definition]{definition}
\declaretheorem[name=Theorem]{theorem}
\declaretheorem[name=Lemma]{lemma}
\declaretheorem[name=Corollary]{corollary}
\declaretheorem[name=Proposition]{proposition}
\declaretheorem[name=Remark]{remark}
\declaretheorem[name=Example]{example}

% ----------------------------------------------------------------------
%  Document metadata - Customize this for your paper!
% ----------------------------------------------------------------------
\title{The Triton Preprint Template}

% Authors with affiliations and contributions
\author[1]{Tianyang Liu}
\author[2,*]{OpenAI o3}
\author[3,*]{Claude Sonnet 4}

% Institution affiliations
\affiliation[1]{University of California San Diego}
\affiliation[2]{OpenAI}
\affiliation[3]{Anthropic}

% Contribution explanations
\contribution[*]{Equal contribution to template writing}
\correspondence{\email{til040@ucsd.edu}}

% Publication date
\date{\today}

% Additional metadata (appears in title block)
\metadata[License]{MIT License}
\metadata[Repository]{\href{https://github.com/leolty/preprint-template}{github.com/leolty/preprint-template}}
\metadata[Version]{v1.0}

% Institution logos (uncomment to display)
% \addlogo[left]{assets/ucsd-logo.png}
% \addlogo[right]{assets/mbzuai-logo.png}

% ----------------------------------------------------------------------
%  Abstract - Describe what this template offers to researchers
% ----------------------------------------------------------------------
\abstract{
The \textbf{Triton} template is a modern, feature-rich \LaTeX{} document class designed specifically for academic preprints in machine learning, computer science, and related fields. This comprehensive guide demonstrates every feature of the template: from basic typography and color schemes to advanced mathematical notation, algorithm formatting, and multi-author collaboration tools. Whether you're writing your first paper or are a seasoned researcher, this template provides everything you need to create professional, beautiful documents that stand out. The template emphasizes clean design, excellent typography, flexible metadata handling, and extensive customization options while maintaining compatibility with major academic publishing workflows.
}


% ======================================================================
\begin{document}
\maketitle
% ======================================================================

\section*{Disclaimer}

\textcolor{ucsdgold}{This document was automatically generated by AI and the content may (and should) contain factual errors, hallucinations, or misleading information. Readers are strongly advised to independently verify all information before relying on or distributing this document.}

\section{Welcome to Triton}

Welcome to \textbf{Triton}\footnote{Named after UC San Diego's mascot, the Triton.}, a comprehensive \LaTeX{} template designed for the modern academic researcher. This template goes far beyond basic document formatting to provide:

\begin{itemize}[leftmargin=15pt]
    \item \textbf{Professional Typography}: Clean, readable fonts optimized for both screen and print
    \item \textbf{Flexible Author Management}: Handle complex authorship with affiliations, contributions, and correspondence
    \item \textbf{Rich Metadata Support}: Add licenses, repositories, versions, and custom metadata
    \item \textbf{Extensive Math Support}: 500+ predefined mathematical commands and symbols
    \item \textbf{Modern Color Palette}: Carefully chosen brand colors that enhance readability
    \item \textbf{Algorithm Formatting}: Beautiful algorithm and code presentation
    \item \textbf{Table \& Figure Tools}: Professional layouts with subcaptions and references
\end{itemize}

This document serves as both a comprehensive feature demonstration and a practical starting point for your own papers.

\section{Getting Started}

\subsection{Quick Start Guide}

Ready to create your first paper with Triton? Follow these simple steps:

\begin{enumerate}[leftmargin=20pt]
    \item \textbf{Copy this template}: Download the template files to your working directory
    \item \textbf{Update metadata}: Edit the title, authors, and affiliations in the preamble
    \item \textbf{Write your content}: Replace the example sections with your research
    \item \textbf{Compile with pdflatex}: Run \texttt{pdflatex paper.tex} to generate your PDF
\end{enumerate}

\subsection{Document Class Options}

Every Triton document begins with the document class declaration:
\begin{tcolorbox}[colback=ucsdgold!8,colframe=ucsdgold,title=Document Class Options]
\begin{verbatim}
\documentclass{triton}              % Single column (default)
\documentclass[twocolumn]{triton}   % Two column layout
\documentclass[preprint]{triton}    % Preprint mode (larger margins)
\end{verbatim}
\end{tcolorbox}

The template automatically loads essential packages and provides sensible defaults, so you can focus on content rather than formatting.

\subsection{Title Block Configuration}

The title block supports rich metadata through simple commands:

\begin{tcolorbox}[colback=ucsdblue!8,colframe=ucsdblue,title=Essential Metadata Commands]
\begin{verbatim}
\title{Your Paper Title}
\author[1,*,†]{First Author}           % Affil, equal contrib, corre.
\author[2,*]{Second Author}            % Multiple affiliations possible
\affiliation[1]{Your Institution}
\contribution[*]{Equal contribution}   % Explain superscripts
\correspondence{\email{you@email.com}}
\date{\today}                          % Or specific date
\end{verbatim}
\end{tcolorbox}

\section{Typography and Visual Design}

\subsection{Professional Color Palette}

Triton provides a carefully curated color palette inspired by UC San Diego's branding:

\begin{center}
\begin{tabular}{@{}lll@{}}
\toprule
\textbf{Color} & \textbf{Usage} & \textbf{Example} \\
\midrule
\textcolor{ucsdnavy}{Navy} & \verb|\textcolor{ucsdnavy}{text}| & Headers, emphasis \\
\textcolor{ucsdblue}{Blue} & \verb|\textcolor{ucsdblue}{text}| & Links, accents \\
\textcolor{ucsdgold}{Gold} & \verb|\textcolor{ucsdgold}{text}| & Highlights, boxes \\
\textcolor{ucsdlightblue}{Light Blue} & \verb|\textcolor{ucsdlightblue}{text}| & Subtle accents \\
\textcolor{ucsdgray}{Gray} & \verb|\textcolor{ucsdgray}{text}| & Secondary text \\
\bottomrule
\end{tabular}
\end{center}

These colors work harmoniously together and maintain excellent readability across different media.

\subsection{Typography Features}

The template includes several typography enhancements that make your papers look professional:

\begin{itemize}
    \item \textbf{Bold text} for strong emphasis
    \item \textit{Italic text} for subtle emphasis  
    \item \texttt{Monospace text} for code and technical terms
    \item \textsc{Small Caps} for acronyms and special terms
    \item \newterm{New term highlighting} with the \verb|\newterm{}| command
\end{itemize}

\section{Mathematical Excellence}

\subsection{Rich Mathematical Notation}

The template ships with over 500 predefined mathematical commands in \texttt{math\_commands.tex}, making it easy to write beautiful equations. Here are some key categories:

\subsubsection{Probability and Statistics}
\begin{align}
P_{\text{error}} &= \E_{\vx \sim \pdata}\left[\1_{f(\vx) \neq y}\right] \\
\KL(\pdata \parallel \pmodel) &= \E_{\vx \sim \pdata}\left[\log \frac{\pdata(\vx)}{\pmodel(\vx)}\right] \\
\mathcal{L}(\vtheta) &= -\E_{(\vx,y) \sim \train}\left[\log \pmodel(y|\vx; \vtheta)\right]
\end{align}

\subsubsection{Linear Algebra}
\begin{align}
\mW &= \begin{bmatrix} \vw_1 & \vw_2 & \cdots & \vw_d \end{bmatrix}^T \\
\|\vx\|_2 &= \sqrt{\sum_{i=1}^n x_i^2} \\
\mSigma &= \E[(\vx - \vmu)(\vx - \vmu)^T]
\end{align}

\subsubsection{Optimization}
\begin{align}
\vtheta^* &= \argmin_{\vtheta} \Ls(\vtheta) \\
\vtheta_{t+1} &= \vtheta_t - \lr \nabla_{\vtheta} \Ls(\vtheta_t) \\
\text{s.t.} \quad &\|\vtheta\|_2 \leq \reg
\end{align}

\subsection{Convenient Math Shortcuts}

The template provides shortcuts for common mathematical objects that save you time:

\begin{table}[htbp]
\centering
\begin{tabular}{@{}lll@{}}
\toprule
\textbf{Type} & \textbf{Command} & \textbf{Output} \\
\midrule
Vectors & \verb|\vx, \vy, \vz| & $\vx, \vy, \vz$ \\
Matrices & \verb|\mA, \mB, \mC| & $\mA, \mB, \mC$ \\
Tensors & \verb|\tX, \tY, \tZ| & $\tX, \tY, \tZ$ \\
Sets & \verb|\sR, \sN, \sZ| & $\sR, \sN, \sZ$ \\
Calligraphic & \verb|\gG, \gH, \gF| & $\gG, \gH, \gF$ \\
Random vars & \verb|\rx, \ry, \rz| & $\rx, \ry, \rz$ \\
\bottomrule
\end{tabular}
\caption{Mathematical notation shortcuts available in the template.}
\end{table}

\section{Theorem Environments}

The template provides a complete set of theorem-like environments that automatically handle numbering and styling:

\begin{definition}[Triton Template]
A \newterm{Triton template} is a \LaTeX{} document class that provides modern, professional formatting for academic preprints with minimal user configuration required.
\end{definition}

\begin{theorem}[Template Completeness]
For any reasonable academic paper structure $\mathcal{S}$, there exists a configuration of the Triton template that can elegantly represent $\mathcal{S}$.
\end{theorem}

\begin{lemma}[Ease of Use]
Learning to use the Triton template effectively requires $O(\log n)$ time where $n$ is the number of features you wish to use.
\end{lemma}

\begin{remark}
The theorem environments automatically handle numbering, cross-references, and consistent styling throughout your document.
\end{remark}

\begin{example}[Quick Start]
To create a new paper, simply copy this template file, update the metadata in the preamble, replace the content sections, and compile with \texttt{pdflatex}.
\end{example}

\section{Algorithm Presentation}

\subsection{Beautiful Algorithm Formatting}

The template provides beautiful algorithm formatting through the \texttt{algorithm} and \texttt{algpseudocode} packages:

\begin{algorithm}[htbp]
\caption{Gradient Descent with Momentum}
\label{alg:momentum}
\begin{algorithmic}[1]
  \Require Learning rate $\lr > 0$, momentum parameter $\beta \in [0,1)$
  \Require Initial parameters $\vtheta_0$, initial velocity $\vv_0 = \vzero$
  \For{$t = 0$ \textbf{to} $T-1$}
    \State Sample mini-batch $\mathcal{B}_t \subset \train$
    \State Compute gradient: $\vg_t = \frac{1}{|\mathcal{B}_t|} \sum_{(\vx,y) \in \mathcal{B}_t} \nabla_{\vtheta} \ell(\vtheta_t; \vx, y)$
    \State Update velocity: $\vv_{t+1} = \beta \vv_t + (1-\beta) \vg_t$
    \State Update parameters: $\vtheta_{t+1} = \vtheta_t - \lr \vv_{t+1}$
  \EndFor
  \State \Return $\vtheta_T$
\end{algorithmic}
\end{algorithm}

\subsection{Code Blocks}

For code snippets, the template provides attractive colored boxes:

\begin{tcolorbox}[colback=ucsdlightblue!10,colframe=ucsdblue,title=PyTorch Implementation Example]
\begin{verbatim}
import torch
import torch.nn as nn

class TritonNet(nn.Module):
    def __init__(self, input_dim, hidden_dim, output_dim):
        super().__init__()
        self.layers = nn.Sequential(
            nn.Linear(input_dim, hidden_dim),
            nn.ReLU(),
            nn.Linear(hidden_dim, output_dim)
        )
    
    def forward(self, x):
        return self.layers(x)
\end{verbatim}
\end{tcolorbox}

\section{Tables and Figures}

\subsection{Professional Tables}

The template encourages the use of \texttt{booktabs} for clean, professional tables:

\begin{table}[htbp]
\centering
\begin{tabular}{@{}lcccc@{}}
\toprule
\textbf{Method} & \textbf{Accuracy} & \textbf{Precision} & \textbf{Recall} & \textbf{F1-Score} \\
\midrule
Baseline & 85.2 & 83.1 & 87.4 & 85.2 \\
Triton-Net & \textbf{92.7} & \textbf{91.5} & \textbf{93.8} & \textbf{92.6} \\
State-of-art & 89.3 & 88.7 & 90.1 & 89.4 \\
\bottomrule
\end{tabular}
\caption{Performance comparison on the benchmark dataset. Bold indicates best results.}
\label{tab:results}
\end{table}

\subsection{Figure Management}

The template works seamlessly with figures and subfigures:

\begin{figure}[htbp]
    \centering
    \begin{subfigure}{0.45\textwidth}
        \centering
        \includegraphics[width=\textwidth]{assets/ucsd-logo.png}
        \caption{UCSD Logo}
        \label{fig:ucsd}
    \end{subfigure}
    \hfill
    \begin{subfigure}{0.45\textwidth}
        \centering
        \includegraphics[width=0.8\textwidth]{assets/mbzuai-logo.png}
        \caption{MBZUAI Logo}
        \label{fig:mbzuai}
    \end{subfigure}
    \caption{Example institutional logos showing the template's figure handling capabilities.}
    \label{fig:logos}
\end{figure}

You can reference individual subfigures (\figref{fig:ucsd} and \figref{fig:mbzuai}) or the entire figure (\figref{fig:logos}) using the convenient reference commands.

\section{Advanced Features}

\subsection{Flexible Lists}

The template supports various list types with customizable spacing:

\begin{enumerate}[leftmargin=20pt]
    \item \textbf{Numbered lists} for sequential information
    \item \textbf{Nested capabilities} for hierarchical content:
    \begin{enumerate}[label=(\alph*)]
        \item Sub-item one
        \item Sub-item two
    \end{enumerate}
    \item \textbf{Custom formatting} through the \texttt{enumitem} package
\end{enumerate}

\subsection{Hyperlinks and References}

The template provides excellent support for cross-references and external links:

\begin{itemize}
    \item Internal references: See \secref{sec:math} for mathematical notation
    \item Algorithm references: \Algref{alg:momentum} shows optimization
    \item Table references: Results are summarized in \Cref{tab:results}
    \item External links: Visit \href{https://github.com/tianyang-liu/triton-template}{the repository} for updates
    \item Email links: Contact the maintainer at \email{til040@ucsd.edu} for support
\end{itemize}

\subsection{Special Text Formatting}

The template includes commands for specialized text formatting:

\begin{quote}
\textit{"The best templates are invisible to the user, allowing them to focus entirely on their content rather than formatting concerns."} 
\end{quote}

You can also create highlighted text boxes for important information:

\begin{tcolorbox}[colback=ucsdgold!15,colframe=ucsdgold,title=Pro Tip]
Use the \verb|\newterm{}| command to consistently highlight newly introduced technical terms throughout your document. This creates visual consistency and helps readers track important concepts.
\end{tcolorbox}

\section{Two-Column Layout}

The template supports both single and two-column layouts. To enable two-column mode, simply pass the \texttt{twocolumn} option to the document class:

\begin{tcolorbox}[colback=ucsdnavy!10,colframe=ucsdnavy,title=Two-Column Mode]
\begin{verbatim}
\documentclass[twocolumn]{triton}
\end{verbatim}
\end{tcolorbox}

This is particularly useful for conference submissions that require two-column formatting.

\section{Best Practices for Academic Writing}

\subsection{Recommended Workflow}

Here's a proven workflow for using the Triton template effectively:

\begin{enumerate}
    \item \textbf{Start with metadata}: Fill in all author information, affiliations, and metadata first
    \item \textbf{Structure first}: Create your section headings and overall document structure
    \item \textbf{Content second}: Focus on writing clear, concise content
    \item \textbf{Formatting last}: Add figures, tables, and fine-tune formatting at the end
\end{enumerate}

\subsection{Customization Options}

The template is designed to be easily customizable. Common modifications include:

\begin{itemize}
    \item Adding new theorem environments with \verb|\declaretheorem|
    \item Customizing colors by redefining the color palette
    \item Adding institution logos with \verb|\addlogo|
    \item Including additional packages for specialized needs
\end{itemize}

\subsection{Performance Tips}

For large documents, consider these optimization strategies:

\begin{itemize}
    \item Use \verb|\includeonly{}| during drafting to compile only specific sections
    \item Keep figures in appropriate formats (PDF for vector graphics, PNG for photos)
    \item Use the \texttt{draft} option during editing to speed up compilation
\end{itemize}

\section{Citations and Bibliography}

The template works seamlessly with standard bibliography tools. For example, foundational works like \citet{lamport1994latex} and \citet{knuth1984texbook} established the typesetting standards we follow today. You can also cite multiple works together \citep{lamport1994latex,knuth1984texbook}.

The template supports various citation styles through the \texttt{natbib} package, making it easy to adapt to different journal requirements.

\section{Getting Help and Support}

\subsection{Common Issues and Solutions}

If you encounter problems, check these common issues:

\begin{itemize}
    \item \textbf{Missing packages}: Ensure you have a complete \TeX{} Live or MiK\TeX{} installation
    \item \textbf{Font issues}: The template uses standard \LaTeX{} fonts; ensure your installation is up-to-date
    \item \textbf{Color problems}: Verify that the \texttt{xcolor} package is properly installed
    \item \textbf{Algorithm formatting}: Make sure both \texttt{algorithm} and \texttt{algpseudocode} packages are available
\end{itemize}

\subsection{Where to Find Help}

If you need assistance:

\begin{enumerate}
    \item Check the template repository for documentation and examples
    \item Review this guide for feature explanations
    \item Open an issue on GitHub for bug reports or feature requests
    \item Contact the maintainer for specific questions
\end{enumerate}

\section{Contributing and Community}

The Triton template is open source and welcomes contributions from the academic community! Ways to get involved include:

\begin{itemize}
    \item Reporting bugs or suggesting improvements
    \item Adding new features or mathematical commands
    \item Improving documentation and examples
    \item Sharing your customizations with other researchers
\end{itemize}

Visit the repository to join the growing community of researchers using modern \LaTeX{} tools.

\section{Conclusion}

The Triton template provides a comprehensive, modern solution for academic writing that balances beautiful design with practical functionality. Whether you're writing your first paper or your hundredth, this template adapts to your needs while maintaining professional standards.

Key benefits of using Triton include:

\begin{itemize}
    \item \textbf{Time savings}: Spend more time on research, less on formatting
    \item \textbf{Professional appearance}: Create documents that look publication-ready
    \item \textbf{Consistency}: Maintain uniform formatting across all your papers
    \item \textbf{Flexibility}: Easily adapt to different journal and conference requirements
    \item \textbf{Community support}: Join a growing community of researchers using modern tools
\end{itemize}

We hope you find the Triton template useful for your academic writing needs. Happy writing!

% Bibliography
\newpage
\bibliographystyle{plainnat}
\bibliography{references}

% ======================================================================
%  APPENDICES
% ======================================================================
\newpage
\beginappendix

\section{Mathematical Command Reference}
\label{sec:math}

This appendix provides a quick reference for the most commonly used mathematical commands available in the template.

\subsection{Vector and Matrix Notation}

\begin{table}[htbp]
\centering
\small
\begin{tabular}{@{}lll@{}}
\toprule
\textbf{Category} & \textbf{Command Examples} & \textbf{Output Examples} \\
\midrule
Vectors & \verb|\vx, \vy, \vz| & $\vx, \vy, \vz$ \\
Matrices & \verb|\mA, \mB, \mC| & $\mA, \mB, \mC$ \\
Tensors & \verb|\tX, \tY, \tZ| & $\tX, \tY, \tZ$ \\
Random vectors & \verb|\rvx, \rvy, \rvz| & $\rvx, \rvy, \rvz$ \\
Matrix elements & \verb|\emA, \emB, \emC| & $\emA, \emB, \emC$ \\
\bottomrule
\end{tabular}
\caption{Vector and matrix notation commands.}
\end{table}

\subsection{Probability and Statistics}

\begin{table}[htbp]
\centering
\small
\begin{tabular}{@{}lll@{}}
\toprule
\textbf{Command} & \textbf{Output} & \textbf{Description} \\
\midrule
\verb|\E| & $\E$ & Expectation operator \\
\verb|\Var| & $\Var$ & Variance operator \\
\verb|\Cov| & $\Cov$ & Covariance operator \\
\verb|\KL| & $\KL$ & KL divergence \\
\verb|\pdata| & $\pdata$ & Data distribution \\
\verb|\pmodel| & $\pmodel$ & Model distribution \\
\bottomrule
\end{tabular}
\caption{Probability and statistics notation.}
\end{table}

\subsection{Optimization and Learning}

\begin{table}[htbp]
\centering
\small
\begin{tabular}{@{}lll@{}}
\toprule
\textbf{Command} & \textbf{Output} & \textbf{Description} \\
\midrule
\verb|\argmax| & $\argmax$ & Argument maximum \\
\verb|\argmin| & $\argmin$ & Argument minimum \\
\verb|\Ls| & $\Ls$ & Loss function \\
\verb|\lr| & $\lr$ & Learning rate \\
\verb|\reg| & $\reg$ & Regularization parameter \\
\verb|\train| & $\train$ & Training dataset \\
\bottomrule
\end{tabular}
\caption{Optimization and learning theory notation.}
\end{table}

\section{Template Customization Examples}

\subsection{Adding Custom Colors}

You can extend the color palette by defining new colors in your preamble:

\begin{tcolorbox}[colback=ucsdgray!10,colframe=ucsdgray,title=Custom Color Definition]
\begin{verbatim}
\definecolor{myred}{RGB}{220,50,47}
\definecolor{mygreen}{RGB}{133,153,0}
\definecolor{myblue}{RGB}{38,139,210}
\end{verbatim}
\end{tcolorbox}

\subsection{Custom Theorem Environments}

Create specialized theorem environments for your research field:

\begin{tcolorbox}[colback=ucsdgray!10,colframe=ucsdgray,title=Custom Theorems]
\begin{verbatim}
\declaretheorem[name=Hypothesis]{hypothesis}
\declaretheorem[name=Conjecture]{conjecture}
\declaretheorem[name=Observation]{observation}
\end{verbatim}
\end{tcolorbox}

\end{document}
